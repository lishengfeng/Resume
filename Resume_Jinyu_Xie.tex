%
% LaTeX source of my resume
% =========================
%
% Heavily commented to to fit even LaTeX beginners (hopefully).
%
% See the `README.md` file for more info.
%
% This file is licensed under the CC-NC-ND Creative Commons license.
%


% Start a document with the here given default font size and paper size.
\documentclass[10pt,a4paper]{article}

% Set the page margins.
\usepackage[a4paper,margin=0.75in]{geometry}

% Setup the language.
\usepackage[english]{babel}
\hyphenation{Some-long-word}

% Makes resume-specific commands available.
\usepackage{resume}

\usepackage{fancyhdr}
\usepackage{lastpage}
 
\pagestyle{fancy}
\renewcommand{\headrulewidth}{0pt}% Remove header rule
\fancyhf{}
% \rhead{jxx123@psu.edu}
% \lhead{Jinyu Xie. Ph.D.}
\rfoot{\thepage \hspace{1pt} of \pageref{LastPage}}

\begin{document}  % begin the content of the document
\sloppy  % this to relax whitespacing in favour of straight margins


% title on top of the document
\maintitle{Jinyu Xie}{(\textbf{Available in Sept. 2016})}{Last update on \today}

\nobreakvspace{0.3em}  % add some page break averse vertical spacing

% \noindent prevents paragraph's first lines from indenting
% \mbox is used to obfuscate the email address
% \sbull is a spaced bullet
% \href well..
% \\ breaks the line into a new paragraph
\noindent\href{mailto:xjygr08@gmail.com}{xjygr08@gmail.com}\sbull
\href{mailto:jxx123@psu.edu}{jxx123@psu.edu}\sbull
\textsmaller{+}1 814-777-5960\sbull
\href{http://sites.psu.edu/jinyuxie/}{sites.psu.edu/jinyuxie}
\\
323 Leonhard Building\sbull
University Park\sbull
PA 16802\sbull
United States

\spacedhrule{0.5em}{-0.8em}  % a horizontal line with some vertical spacing before and after

\roottitle{Summary}  % a root section title

Jinyu's research interest has been focused on: (1) \textbf{Dynamic System Modeling and System
Identification} (Data-driven modeling, Time Series Analysis); (2) \textbf{Control
Theories} (Optimal Control, Predictive Control); (3) \textbf{Signal Processing}
(State/Parameter Estimation, Fault/Maneuver Detection). \vspace{0.3em} 

\noindent Passionate about intelligent systems aiming at creating better world with
multiple projects:

(1) Research Project: Autonomous Insulin Delivery System (Artificial Pancreas) for Type
1 Diabetes Treatment;

(2) Intern Project: Pipeline Oil Leak Monitoring and Detection System, Shell Oil Company;

(3) Could take place in your team.


\spacedhrule{0.5em}{-0.8em}

\roottitle{Education}

\headedsection
{Pennsylvania State University, University Park, PA}
{GPA 3.91/4.00}{
  \headedsubsection
  {Ph.D. Mechanical Engineering (Major), Computational Science (Minor)}
  {Aug 2012 -- present}{}

  \headedsubsection
  {M.S Mechanical Engineering}{Aug 2012 -- May 2016}{}
}

\headedsection
{Tsinghua University, Beijing, China}{GPA 3.70/4.00}{
	\headedsubsection
  {B.S. Instrumentation and Controls Engineering}{Aug 2008 -- July 2012}{}
}

\spacedhrule{0.5em}{-0.8em}
\roottitle{Skills}
% \begin{center}
\small{
\begin{tabular}{ll}
  
  {\normalsize{\textbf{Programming: }}} & Matlab/Simulink (\textit{daily use, 5 years}), \textit{C/C++ (course projects/campus competitions}), \LaTeX (\textit{daily use}),\\
                                        & Version Control (\textit{Git, daily use}), Linux/Unix Shell Command Line (\textit{Basic Knowledge})\\
  
  {\normalsize \textbf{Industrial Software: }} & PI Processbook (\textit{intern daily use}),
                                                 Shell Production Universe (\textit{intern daily use}),\\
                                        & AutoCAD/Solidworks (\textit{course projects})\\
  {\normalsize{\textbf{Technical Expertise: }}} & Dynamic Systems and Control Theories (\textit{PID, State Space, Predictive Control, 6 relevant publications})\\
                                        & Signal Processing (\textit{Extended Kalman Filter, FIR Filter, Fault Detection, 4 relevant publications/internship})\\
                                        & Time Series Analysis/Modeling (\textit{AR, ARX, ARMAX, Box-Jenkins, 2 relevant publications})\\
                                        & Machine Learning (\textit{LDA, QDA, KNN, CART, K-means, Mixture Model, course projects})
\end{tabular}
}
% \end{center}

% \inlineheadsection
% {Programming:}{Matlab/Simulink (daily use, 5 years), C/C++ (course
%   projects/campus competitions), Version Control (Git, daily use), \LaTeX (daily use), Linux/Unix Shell Command Line (Basic Knowledge)}

% \inlineheadsection
% {Industrial Software (Internship Daily Use):}{PI Processbook,
%   Production Universe (Shell Oil\&Gas)}

% \inlineheadsection
% {Control Systems Design (2 relevant publications):}{Classical Control Theories
% (Laplace Transform, Bode/Nyquist Plot, PID Loop Tuning), Modern Control Theories
% (State Space, Model-Based Optimal Control, Linear Quadratic Regulator, Robust
% Control, Model Predictive Control)}

% \inlineheadsection
% {Signal Processing (6 relevant publications):}{(Extended)
% Kalman Filter, Finite Impulse Response (FIR) Filter, Fault Detection and Estimation}

% \inlineheadsection
% {Statistics (6 relevant publications):}{Time Series Analysis/Modeling (AR, ARX,
%   ARMAX, Box-Jenkins), Estimation Theories, Hypothesis Test}

% \inlineheadsection
% {Machine Learning (Course Projects):}{Principal Component Analysis,
%   Supervised Learning (Linear/Quadratic Discriminant Analysis, Logistic
%   Regression, CART, K-Nearest-Neighbor), Unsupervised Learning (K-means, Mixture
%   Models)}

% \inlineheadsection
% {3D Modeling/FEA (Course Projects):}{AutoCAD, Solidworks, COMSOL}

\spacedhrule{0.5em}{-0.8em}

\roottitle{Work Experience}

\headedsection
{\href{http://sites.psu.edu/jinyuxie/research/}{Information \& Controls Lab, Pennsylvania State University}}
{\textsc{University Park, PA}}
  {
    \headedsubsection
    {Research Assistant - Control Algorithm Development for Artificial Pancreas}
    {Aug 2012 -- present}
    {
      \begin{itemize}
      \item NSF Funded Project with 3 journals and 3 conference proceedings:
        Modeling, Identification, and Control towards Adaptive Personalized Glucose
        Management for Insulin-Deficient Diabetes.
        
      \item Designed data collection experiment protocols that are both
        clinically practical and highly excited for system identification.

      \item Proposed a nonlinear data-driven model with multiple physiological
        inputs to capture the blood sugar dynamics of the diabetic patients.
        Validated the model with large scale of time-series data (3-day-length
        data from 30 virtual patients and 5 real patients). 
        
      \item Applied signal processing algorithms (Extended Kalman Filter, FIR
        Filter) and time series analysis (ARMAX) to data preprocessing, model
        parameter estimation and state estimation. Predicted the blood sugar level in 30
        minutes with accuracy as high as 85\% in terms of FIT values.

      \item Developed an Adaptive Model Predictive Controller that can
        ``intelligently learn'' the model of the patient in real time and deliver
        personalized therapy regimen.
        
      \item Introduced an innovative adaptive filtering algorithm (Variable
        State Dimension Filter) to detect and estimate the meals taken by
        patients with a successful detection rate of 96\% and false alarm rate
        of 8\%.
      \end{itemize}
    }
  }

  \headedsection
  {\href{http://sites.psu.edu/jinyuxie/internship-project/}{Process Automation Control \& Optimization Group, Shell Oil Company}}
  {\textsc{Houston, TX}}
  {
    \headedsubsection
    {Deepwater R\&D Engineer, Post Grad Intern}{May 2015 -- Aug 2015}
    {
      \begin{itemize}
      \item Delivered an enhanced pipeline leak detection system together with
        technical reports ready for commission and deployment in time and
        quality.
        
      \item Highlighted intern personnel by a General Manager at Shell US PhD
        Intern Symposium. Received a prestigious award nominated by Process
        Automation Control \& Optimization Group. 
        
      \item Worked in a project team (3 people across 2 different functionality
        groups). Weekly meet up for brain storming and task distribution. Biweekly
        report to project managers and line managers. Exposure to risk management,
        algorithm prototyping (using Matlab) and hands-on implementation (using PI
        Processbook and Shell Production Universe).
        
      \item Enhanced the existing pipeline leak monitor system (only detect
        oil leakage in steady-state flow) to detect leakage during noisy
        transient states by introducing model-based fault detection
        methodologies. The enhanced monitor system covers the full operation time
        (100\%) compared with 60\% of time with the original monitor system.  
        
      \item Identified the bias of the existing estimation algorithm by going
        through 2 years of plant data, and presented a compensation strategy that
        achieved 5\% improvement of estimation accuracy.
        
      \item Sped up the system deployment and replication procedure by modifying the system
        structures and eliminating redundant codes (Visual Basic).
        
      \item Prepared simulation demo videos for operator education purposes.
      \end{itemize}
    }
  }

  \spacedhrule{0.5em}{-0.8em}

  \roottitle{Course Projects}
  \headedsection
  {Machine Learning -- Satellite Image Semantics Classification}{\textsc{State
      College, PA}}{
    \headedsubsection{Team Size: 2}{Aug 2014 -- Dec 2014}{
      \bodytext{Implemented (self implementation \& package used for verification) multiple classification and clustering algorithms
        and performed k-fold cross validation on a data set containing 2400 images of
        8x8 pixels labeled with 14 semantic categories (beach, mountain, etc.). Improved
        the prediction performance (running time \& prediction accuracy) by leveraging
        Principal Component Analysis (PCA). }
    }
  }

  \headedsection
  {Mechatronic Car -- Maze Runner}{\textsc{Beijing, China}}{
    \headedsubsection
    {Team Size: 4}{Feb 2012 -- May 2012}
    {\bodytext{Designed and assembled a smart car that searches feasible
        paths towards target zones in a maze. My work included sensor arrays
        alignment and installation, Printed Circuit Board (PCB) design of the
        power system, embedded C programming of its motor control algorithm and
        path search strategies on a microprocessor (MSP430). 
      }
    }
  }

  \headedsection
  {Shopping Assistant Robot}{\textsc{Beijing, China}}
  {
    \headedsubsection
    {Team Size: 4}{Aug 2011 -- Dec 2011}{
      \bodytext{Designed and assembled a robot that assists the disabled shopping in
        supermarkets. I am responsible for the grasping mechanism of the robot. My
        work included kinematic analysis, material strength check, technical drawing
        (AutoCAD, Solidworks), manufacturing process design, assembling and testing.
        Runner-up in a campus mechanical design competition.}
    }
  }

  \headedsection
  {\textit{REVERTI} Game Programming (C)}{\textsc{Beijing, China}}
  {
    \headedsubsection
    {Team Size: 1}{Feb 2009 -- May 2009}{
      \bodytext{A strategy board game for two players programmed by C.
        Functionalities included player registration and log in, ranking systems, and a simple
        artificial intelligence (AI).
      }
    }
  }
  
%%%%%%%%%%%%%%%%%%%%%%%%%%%%%%%%%%%%%%
  % \spacedhrule{0.5em}{-0.8em}
  % \roottitle{Academia Activities}
    
  % \textbf{Membership,} American Society of Mechanical Engineering
  %   (ASME).\\
  % \textbf{Invited Reviewer (2016), Session Organizer (2015)}.
  % ASME Dynamic Systems and Control Conference.
  
  
%%%%%%%%%%%%%%%%%%%%%%%%%%%%%%%%%%%%%%
  \spacedhrule{0.5em}{-0.5em}
  \roottitle{Honors}
  \begin{itemize}
  \item \textbf{Vantage Award - Above \& Beyond}. Process
    Automation Control \& Optimization Group, Shell Oil Company
  \item \textbf{Session Organizer (2015), Invited Reviewer(2016).} ASME Dynamic Systems
    and Control Conference
  \item \textbf{Student Travel Award:} 2015 American Control Conference, 2015 Dynamic Systems and Control Conference
  \end{itemize}

%%%%%%%%%%%%%%%%%%%%%%%%%%%%%%%%%%%%%%
  % \spacedhrule{2em}{-0.5em}

  % \roottitle{Courses}
  
  % \begin{center}
  %   \small
  %   \begin{tabular}{l l}
  %     \textbf{\normalsize Mathematics:} & Real Analysis, Theory of Probability, Theory of Statistics, Numerical Optimization \\
  %                           & Numerical Linear Algebra, Stochastic Process and Monte Carlo Methods, Nonlinear Programming\\
  %     \textbf{\normalsize Computer Science: } & Advanced Computer Programming (C++), Machine Learning, Digital Signal Processing\\
  %     \textbf{\normalsize System Control:} & Linear System, Nonlinear Control, Optimal Control, Robust Control
  %   \end{tabular}
  % \end{center}

  \spacedhrule{1em}{-0.8em}
  \roottitle{Technical Talks}
  \begin{itemize}
  \item Meal Detection and Estimation for Type 1 Diabetes: A
      Variable State Dimension Approach. \textit{Oct 2015. 2015 Dynamic Systems \& Control
      Conference (DSCC), Columbus, OH.}

  \item Model Predictive Control for Type 1 Diabetes Based on Personalized LTV
    Model with Insulin and Meal Inputs. \textit{July 2015. 2015 American Control
      Conference (ACC), Chicago, IL}
    
  \item Fault Detection Based on Modeling and Estimation Methods. \textit{July 2015.
    Shell Deepwater R\&D Group Lunch \& Learn Seminar. Houston, TX.}
  \end{itemize}

\spacedhrule{0.5em}{-0.8em}
\roottitle{Publications}
  
  \begin{itemize}
  \item \textbf{Jinyu Xie}, and Qian Wang. ``A Nonlinear Data-Driven Model of
    Glucose Dynamics Accounting for Physical Activity for Type 1 Diabetes: An in
    Silico Study.'' \textit{Accepted. ASME 2016 Dynamic Systems and Control
    Conference.} 
  \item \textbf{Jinyu Xie}, and Qian Wang. ``A Variable State Dimension Approach
    to Meal Detection and Meal Size Estimation: in Silico Evaluation through
    Basal-Bolus Insulin Therapy for Type 1 Diabetes.'' \textit{Under review. IEEE
    Transactions on Biomedical Engineering.}
  \item \textbf{Jinyu Xie}, and Qian Wang. ``Meal Detection and Meal Size Estimation for Type 1 Diabetes Treatment: A Variable State Dimension Approach.'' \textit{In ASME 2015 Dynamic Systems and Control Conference, pp. V001T15A003. American Society of Mechanical Engineers, 2015.}
  \item Qian Wang, \textbf{Jinyu Xie}, et al. ``Model Predictive Control for Type 1 Diabetes Based on Personalized Linear Time-Varying Subject Model Consisting of Both Insulin and Meal Inputs An in Silico Evaluation'' \textit{Journal of diabetes science and technology (2015): 1932296815586426.}
  \item Qian Wang, \textbf{Jinyu Xie}, et al. ``Model Predictive Control for Type 1 Diabetes Based on Personalized Linear Time-Varying Subject Model Consisting of both Insulin and Meal Inputs: an in Silico Evaluation.'' \textit{American Control Conference, 2015: 5782-5787.}
  \item Qian Wang, Peter Molenaar, Saurabh Harsh, Kenneth Freeman, \textbf{Jinyu Xie}, et al. ``Personalized State-space Modeling of Glucose Dynamics for Type 1 Diabetes Using Continuously Monitored Glucose, Insulin Dose, and Meal Intake An Extended Kalman Filter Approach.'' \textit{Journal of diabetes science and technology 8.2 (2014): 331-345.}
  \end{itemize}
  
  
\end{document}

＀