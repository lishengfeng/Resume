%
% LaTeX source of my resume
% =========================
%
% Heavily commented to to fit even LaTeX beginners (hopefully).
%
% See the `README.md` file for more info.
%
% This file is licensed under the CC-NC-ND Creative Commons license.
%


% Start a document with the here given default font size and paper size.
\documentclass[10pt,a4paper]{article}

% Set the page margins.
\usepackage[a4paper,margin=0.5in]{geometry}

% Setup the language.
\usepackage[english]{babel}
\hyphenation{Some-long-word}

% Makes resume-specific commands available.
\usepackage{resume}

\usepackage{fancyhdr}
\usepackage{lastpage}
 
\pagestyle{fancy}
\renewcommand{\headrulewidth}{0pt}% Remove header rule
\fancyhf{}
% \rhead{jxx123@psu.edu}
% \lhead{Jinyu Xie. Ph.D.}
\rfoot{\thepage \hspace{1pt} of \pageref{LastPage}}

\begin{document}  % begin the content of the document
\sloppy  % this to relax whitespacing in favour of straight margins


% title on top of the document
\maintitle{Jinyu Xie, Ph.D.}{}{Last update on \today}

\nobreakvspace{0.3em}  % add some page break averse vertical spacing

% \noindent prevents paragraph's first lines from indenting
% \mbox is used to obfuscate the email address
% \sbull is a spaced bullet
% \href well..
% \\ breaks the line into a new paragraph
\noindent\href{mailto:xjygr08@gmail.com}{xjygr08@gmail.com}\sbull
814-777-5960\sbull\href{https://www.linkedin.com/in/jinyuxie/}{www.linkedin.com/in/jinyuxie}\sbull
\href{https://github.com/jxx123}{github.com/jxx123}

\spacedhrule{0.5em}{-0.8em}  % a horizontal line with some vertical spacing before and after
\roottitle{Summary}  % a root section title
\begin{itemize}
\setlength{\itemsep}{0.1em}
\item 5+ years domain knowledge in \textbf{Fault Diagnosis and Prognostics, Machine Learning, Time Series Analysis, Signal Processing, and Control Systems} across academia research and industrial applications. 
\item 5+ years of experience in \textbf{Matlab/Simulink}, 2 years of experience
  in \textbf{Python}, working knowledge in
  \textbf{C/C++}.
 \item Exposure to professional software development cycle (API design, version control, peer code review, unit test) 
\item  R\&D Experience across \textbf{Biomedical, Oil \& Gas, Software Industries}.
\end{itemize}
% \roottitle{Objective}
% To obtain an full-time position that
% fits my background of \textbf{Control Systems, Signal
% Processing (State Estimation / Fault Detection) and Computational Science}. Available in \textbf{September 2016}. 
% \roottitle{Summary}  % a root section title
% $\bullet$ Jinyu's research interest has been focused on \textbf{Dynamic System
%   Modeling, System Identification, Signal Processing and Control Theories} with
% 6 related publications.

% \noindent $\bullet$ Extensive knowledge of \textbf{Matlab/Simulink} in research and fast
% prototyping (Doctoral Research, 4 years). \vspace{0.2em}

% \noindent $\bullet$ Experience of \textbf{Machine Learning} and \textbf{C/C++} with
% multiple projects (Image Classification, Robotics, Game).\vspace{0.2em}

% \noindent $\bullet$ Industrial Research \& Development Experience of \textbf{Signal
% Processing and Fault Detection} at Shell Oil Company.\vspace{0.2em}



% \inlineheadsection
% {Programming:}{Matlab/Simulink (daily use, 5 years), C/C++ (course
%   projects/campus competitions), Version Control (Git, daily use), \LaTeX (daily use), Linux/Unix Shell Command Line (Basic Knowledge)}

% \inlineheadsection
% {Industrial Software (Internship Daily Use):}{PI Processbook,
%   Production Universe (Shell Oil\&Gas)}

% \inlineheadsection
% {Control Systems Design (2 relevant publications):}{Classical Control Theories
% (Laplace Transform, Bode/Nyquist Plot, PID Loop Tuning), Modern Control Theories
% (State Space, Model-Based Optimal Control, Linear Quadratic Regulator, Robust
% Control, Model Predictive Control)}

% \inlineheadsection
% {Signal Processing (6 relevant publications):}{(Extended)
% Kalman Filter, Finite Impulse Response (FIR) Filter, Fault Detection and Estimation}

% \inlineheadsection
% {Statistics (6 relevant publications):}{Time Series Analysis/Modeling (AR, ARX,
%   ARMAX, Box-Jenkins), Estimation Theories, Hypothesis Test}

% \inlineheadsection
% {Machine Learning (Course Projects):}{Principal Component Analysis,
%   Supervised Learning (Linear/Quadratic Discriminant Analysis, Logistic
%   Regression, CART, K-Nearest-Neighbor), Unsupervised Learning (K-means, Mixture
%   Models)}

% \inlineheadsection
% {3D Modeling/FEA (Course Projects):}{AutoCAD, Solidworks, COMSOL}

\spacedhrule{0em}{-0.8em}

\roottitle{Work Experience}

  \headedsection
{Design Optimization \& Identification Group, MathWorks Inc.}
{\textsc{Natick, MA}}
{
  \headedsubsection
  {Software Engineer - Predictive Maintenance Toolbox}
  {April 2017 - Present}
  {
    \begin{itemize}
    \item Explore and develop Feature Extraction and Selection algorithms for Predictive Maintenance purpose
    \item Develop tools to perform Spectrum Analysis on vibration signal to detect and isolate faults
    \item Apply Machine Learning algorithms for fault classification
    \item Design Graphical User Interface (GUI) for Predictive Maintenance workflows
    \end{itemize}
  }
}

\headedsection
{Information \& Controls Lab, Pennsylvania State University}
{\textsc{University Park, PA}}
  {
    \headedsubsection
    {Research Assistant - Glucose Monitoring \& Control Algorithm Development}
    {Aug 2012 -- May 2017}
    {\vspace{0.1em}
      \begin{itemize}
        \setlength{\itemsep}{0.3em}
      \item Designed system identification experiments
        to collect informative clinical data (35 Patients, 3-day-scenario).

      \item Identified a nonlinear data-driven model for blood glucose metabolic
        systems that significantly improved the long-term glucose prediction by adding bilinear features with physiological insights.
      
      \item Developed a personalized dietary and exercise recommender system to
        enhance diabetic patients' self management and minimize the clinical risk.  
      \item Proposed an innovative algorithm (Variable State Dimension
        Algorithm) to detect and estimate unexpected maneuver (meals/exercise) with a
        sensitivity of 96\% and false alarm rate of 8\%.
        
      % \item Predicted the blood glucose level with a fit of 85\% by implementing
      %   advanced signal processing/estimation algorithms (Time Series, Extended Kalman Filter, FIR
      %   Filter).
        
        % Predicted the blood glucose level in 30
        % minutes with accuracy as high as 85\% in terms of FIT values.

      \item Synthesized an Adaptive Model Predictive Control Algorithm to estimate
        model parameters online and deliver personalized optimal insulin therapy
        (outperformed a PID controller by 40\% in terms of risk indices).
        % \item NSF Funded Project:
      %   Modeling, Identification, and Control towards Adaptive Personalized Glucose
      %   Management for Insulin-Deficient Diabetes. Outcome: 3 journals and 3 conference proceedings.
        
      % \item Designed data collection experiment protocols that are both
      %   clinically practical and highly excited for system identification.

      % \item Proposed a nonlinear data-driven model with multiple physiological
      %   inputs (Insulin, Meal, Exercise) to capture the blood glucose dynamics of the
      %   diabetic patients. Validated the model with large scale of time-series data.
        % (3-day-length data from 30 virtual patients and 5 real patients).
       
      % \item Designed system identification protocols
      %   to collect informative clinical data (35 Patients, 3-day-scenario).

      % \item Identified a nonlinear multivariate data-driven model for 
      %   blood glucose metabolic systems that captured the effects of insulin,
      %   carbohydrate and physical activity correspondingly by performing
      %   Time Series Analysis (ARMAX) and State/Parameter Estimation (Extended Kalman Filter).
      
      % \item Improved the long-term glucose prediction (3 to 5 hours ahead) in
      %   terms of $R^{2}$ and Clark Error Grid Analysis by adding a bilinear feature to
      %   describe the insulin sensitivity change induced by physical activity.
        
      % \item Proposed an adaptive filter (Variable State Dimension
      %   Approach based on Kalman Filter) to detect and estimate unexpected maneuver
      %   (meals/exercise) with a sensitivity of 96\% and false alarm rate of 8\%.
        
      % % \item Predicted the blood glucose level with a fit of 85\% by implementing
      % %   advanced signal processing/estimation algorithms (Time Series, Extended Kalman Filter, FIR
      % %   Filter).
        
      %   % Predicted the blood glucose level in 30
      %   % minutes with accuracy as high as 85\% in terms of FIT values.

      % \item Developed an Adaptive Model Predictive Control Algorithm to learn the
      %   patient model online and deliver personalized optimal insulin therapy
      %   (outperformed a PID controller by 40\% in terms of risk indices).
       
      % \item Solved numerical optimization problems in real time (Simulink) with
      %   open source optimization packages (Yalmip, SeDuMi, SDPT3).  
      % \item Publish and present results in peer-reviewed conferences and
      %   journals (3 conferences and 3 journals published).
      % \item Introduced an innovative adaptive filtering algorithm (Variable
      %   State Dimension Filter) to detect and estimate the meals taken by
      %   patients with a successful detection rate of 96\% and false alarm rate
      %   of 8\%.
      \end{itemize}
    }
  }
  
  \headedsection
{Process Data Technology Group, Air Products Inc.}
{\textsc{Allentown, PA}}
{
  \headedsubsection
  {PhD Intern - Predictive Modeling and Supply Chain Optimization}
  {Sep 2016 - Dec 2016}
  {
    \begin{itemize}
    \item Identified hydrogen refinery models (Box-Jenkins Models, 17 inputs, 36 outputs) with closed-loop data.
      
    % \item Perform sequential experimental designs (Adaptive D-Optimal) for
    %   supply chain model identification.
      
    \item Delivered APIs for sequential design of experiment (adaptive D-Optimal) and supply chain optimization (mixed integer programming, optimization surface estimation).
      
    % \item Identify hydrogen refinery process models (Box-Jenkins Models, 17 inputs,
    %   36 outputs) using closed-loop data.
      
    % \item Perform sequential experimental designs (Adaptive D-Optimal) for
    %   supply chain model identification.
    % % \item Implement Model Predictive Controllers and Minimum Variance Controllers.
    \end{itemize}
  }
}


\vspace{0.2em}
  \headedsection
  {Process Automation Control \& Optimization Group, Shell Oil Company}
  {\textsc{Houston, TX}}
  {
    \headedsubsection
    {PhD Intern - Signal Processing \& Fault Detection}{May 2015 -- Aug 2015}
    {\vspace{0.1em}
      \begin{itemize}
        \setlength{\itemsep}{0.3em}
      \item Delivered an enhanced pipeline leak detection system 
        ready for commission and deployment.
       
      \item Analyzed large scale of plant data (2 million records) and
        identified fault signatures and root causes.
       
      \item Built data-driven models to estimate pipeline flow rates under
        limited instrumentation with error rate $< 5\%$.
        
      \item Raised the system uptime from 60\% to 100\% with false alarm rate
        less than 1\% by applying advanced signal processing algorithms (model-based
        fault detection methodologies).
        
      % \item Worked in a project team (3 people across 2 different functionality
      %   groups). Weekly meet up for brain storming and task distribution. Biweekly
      %   report to project managers and line managers. Exposure to risk management,
      %   algorithm prototyping and hands-on implementation.
        
      % \item Enhanced the pipeline leak monitor system to detect leakage during noisy transient states by
      %   applying advanced signal processing algorithms (model-based fault detection).
      %   The system uptime was raised from 60\% to 100\%.
        
      % \item Identified the bias of the existing estimation algorithm by going
      %   through 2 years of plant data, and presented a compensation strategy that
      %   achieved 5\% improvement of estimation accuracy.
        
      \item Sped up the system deployment and replication procedure by modifying
        the system structures and eliminating redundant codes (Visual Basic).
             
      % \item Prepared reading materials and simulation demos for operator
      %   education purposes.
        
      % \item % Highlighted intern personnel by a General Manager at Shell US
      %   % Graduate Intern Symposium.
      %   Received a prestigious award nominated by Process
      %   Automation Control \& Optimization Group.
        
      \end{itemize}
    }
  }

% \vspace{0.2em}
%   \headedsection
%   {Robotics Lab, Tsinghua University}{\textsc{Beijing, China}}
%   {
%     \headedsubsection
%     {Research Assistant - Autonomous Car Circuit Design and Embedded Programming}{Sep 2010 -- Jan 2012}{
%       \vspace{0.1em}
%       \begin{itemize}
%         \setlength{\itemsep}{0.3em}
%       % \item Single-legged Hopping Robot
%       %  \begin{itemize} 
%       % \item Performed kinematic analysis and balance analysis of the
%       %   single-legged hopping robot
        
%       % \item Tuned PID controller of an air pump serving as the actuator of the robot.
%       % \end{itemize}

%       \item Designed and assembled an Autonomous Car searching feasible paths
%         towards target zones in a maze. % The car reached 15 random generated target zones
%         % within 3 minutes, and won a second prize in a campus competition.
%       % \item Aligned and installed optical sensor arrays to optimize the
%       %   detectability of working environment.
%       \item Designed a Printed Circuit Board to distribute power into 
%         Optical Sensors, Microchip (MSP430) and DC Motors.
%       \item Programmed (Embedded C) the control algorithms (Pulse-Width-Modulation,
%         PID) for the DC Motors and a path planning strategy (Greedy Algorithm) in a Microprocessor (MSP430).
%       \item The car reached 15 random generated target zones in a maze within 3 minutes.

%     % \item Robot Assisting Shopping for the Handicap
%     %   \item Designed a grasp mechanism that is able to grasp various shapes of
%     %     items.
        
%     %   \item Conducted kinematic analysis, strength check, technical
%     %     drawing (AutoCAD/Solidworks), manufacturing process design and
%     %     assembling.
        
%     %   \item The robot conquered all the challenges within 5 minutes and won a
%     %     runner-up in a campus robot competition.
% \end{itemize}
%     }
%   }
%   \headedsection
%   {Machine Learning Project}{\textsc{State
%       College, PA}}{
%     \headedsubsection{Satellite Image Semantics Classification}{Aug 2014 -- Dec 2014}{
%       \begin{itemize}
%       \item Implemented multiple classification and clustering algorithms using Python
%         (Logistic Regression, Neural Network, LDA, QDA, Decision Tree, K-Means,
%         EM Algorithm).
        
%       \item Performed cross validation on a data set containing 2400 images of
%         8x8 pixels labeled with 14 semantic categories (beach, mountain, etc.)
%         with a classification accuracy of 52.17\% (Neural Network, backpropogation).

%       \item Reduced the computation cost by 30\% with minor degradation of
%         prediction accuracy ($<0.1\%$) by performing Principal Component
%         Analysis (PCA) (dimension reduced from 64 to 35 with 98\% variance preserved).
%       \end{itemize}
%       }
%      }

\spacedhrule{0em}{-0.8em}

\roottitle{Education}

\headedsection
{Pennsylvania State University, University Park, PA}
{GPA 3.91/4.00}{
  % \headedsubsection
  % {Ph.D. Mechanical Engineering}
  % {Aug 2012 -- Aug 2016 (Expected)}{}

  \headedsubsection
  {Ph.D. Mechanical Engineering (Major), Computational Science (Minor)}{Aug 2012
    -- May 2017}{}
  \headedsubsection
  {M.S. Mechanical Engineering}{Aug 2012
    -- May 2016}{\vspace{0.3em}\bodytext{\textbf{Ph.D. Thesis:} Intelligent Artificial Pancreas Incorporated with Maneuver Detection and Recommender System for Type-1 Diabetes Self Management
      }}
      % \vspace{0.3em}\\
      % \textbf{$\bullet$ Relevant Courses:} Real Analysis, Numerical Optimization, Theory of
      % Probability/Statistics, Digital Signal Processing, Classical/Linear/Nonlinear/Optimal/Robust Control, C++
      % Programming, Machine Learning}}
}
\vspace{0.1em}
\headedsection
{Tsinghua University, Beijing, China}{GPA 3.70/4.00}{
	\headedsubsection
  {B.S. Mechanical Engineering}{Aug 2008 -- July 2012}{}
}


\spacedhrule{0.1em}{-0.8em}
\roottitle{Skills}
Matlab/Simulink, Python, C/C++, Version Control (Git), Linux/Unix Shell, Machine
Learning, Time Series Analysis, Signal Processing, Control Systems
% \begin{tabular}{ll}
  
%   \textbf{Programming: } & Matlab/Simulink, Python, C/C++, Version Control (Git), Linux/Unix Shell\\
%   \textbf{Others: } & Machine Learning (PCA, Logistic Regression, Neural Network, SVM, K-means, EM Algorithm), \\
%   & Digital Signal Processing (FIR/IIR Filter), Control Systems (Kalman Filter, PID, LQR, LQG, MPC)

%   % \textbf{Hardware: } & MSP430, FPGA\\ 
%   % \textbf{Numerical Optimization: } & YALMIP, CPLEX, SeDuMi, SDPT3\\
%   % \textbf{Industrial Software: } & PI Processbook, UniSim (Honeywell), AutoCAD, Solidworks, COMSOL, Altium\\ 
%   % \textbf{Machine Learning: } & Dimensionality Reduction (PCA), Logistic Regression, Neural Network,\\
%   %                               &  Support Vector Machine, Decision Tree, Random Forest,  K-means, EM Algorithm 
%   % {\normalsize{\textbf{Control Systems: }}} & PID, State Space, Optimal Control, Robust Control, Predictive Control. (\textit{Master Thesis})\\

%   % {\normalsize{\textbf{Signal Processing: }}} & FIR Filter, Extended Kalman Filter, Fault Detection/Estimation. (\textit{Master Thesis + Internship})\\

%   % {\normalsize{\textbf{Machine Learning: }}} & Dimensionality Reduction, Regression, Classification, Clustering. (\textit{Master Thesis + Course Projects})
%   % {\normalsize{\textbf{Technical Expertise: }}} & Dynamic Systems and Control Theories (\textit{PID, State Space, Predictive Control, 6 relevant publications})\\
%   %                                       & Signal Processing (\textit{Extended Kalman Filter, FIR Filter, Fault Detection, 4 relevant publications/internship})\\
%   %                                       & Data Analysis (Time Series, Regression, Classification, Clustering, Hypothesis Test, 2 relevant publications and intern projects)\\
%   %                                       & Time Series Analysis/Modeling (\textit{AR, ARX, ARMAX, Box-Jenkins, 2 relevant publications})\\
%   %                                       & Machine Learning (\textit{LDA, QDA, KNN, CART, K-means, Mixture Model, course projects})
% \end{tabular}

% %%%%%%%%%%%%%%%%%%%%%%%%%%%%%%%%%%%%%%
%   % \spacedhrule{0.5em}{-0.8em}
%   % \roottitle{Academia Activities}
    
%   % \textbf{Membership,} American Society of Mechanical Engineering
%   %   (ASME).\\
%   % \textbf{Invited Reviewer (2016), Session Organizer (2015)}.
%   % ASME Dynamic Systems and Control Conference.
  
  

% %%%%%%%%%%%%%%%%%%%%%%%%%%%%%%%%%%%%%%
%   % \spacedhrule{2em}{-0.5em}

%   % \roottitle{Courses}
  
%   % \begin{center}
%   %   \small
%   %   \begin{tabular}{l l}
%   %     \textbf{\normalsize Mathematics:} & Real Analysis, Theory of Probability, Theory of Statistics, Numerical Optimization \\
%   %                           & Numerical Linear Algebra, Stochastic Process and Monte Carlo Methods, Nonlinear Programming\\
%   %     \textbf{\normalsize Computer Science: } & Advanced Computer Programming (C++), Machine Learning, Digital Signal Processing\\
%   %     \textbf{\normalsize System Control:} & Linear System, Nonlinear Control, Optimal Control, Robust Control
%   %   \end{tabular}
%   % \end{center}


%   \roottitle{Projects}
%   \headedsection
%   {Machine Learning}{\textsc{State
%       College, PA}}{
%     \headedsubsection{Satellite Image Semantics Classification, Team Size: 4}{Aug 2014 -- Dec 2014}{
%       \begin{itemize}
%       \item Implemented multiple classification and clustering algorithms using Python
%         (Logistic Regression, Neural Network, LDA, QDA, Decision Tree, K-Means,
%         EM Algorithm).
        
%       \item Performed 4-Fold Cross Validation on a data set containing 2400 images of
%         8x8 pixels labeled with 14 semantic categories (beach, mountain, etc.)
%         with a classification accuracy of 52.17\% (Neural Network, 2 hidden
%         layers, 1024 Neuron Units in each layer).

%       \item Reduced the computation cost by 30\% with minor degradation of
%         prediction accuracy ($<0.1\%$) by performing Principal Component
%         Analysis (PCA) (dimension reduced from 64 to 35 with 98\% variance preserved).
%       \end{itemize}
%       }
%      }
 

% %   \headedsection
% %   {Mechatronics -- A Car Searching Path in a Maze}{\textsc{Beijing, China}}{
% %     \headedsubsection
% %     {Team Size: 4}{Feb 2012 -- May 2012}
% %     {\bodytext{Designed and assembled a smart car that searches feasible paths
% %         towards target zones in a maze. My work included sensor arrays alignment, PCB
% %         design of the power system, embedded C programming of the motor control
% %         algorithm (Pulse-Width-Modulation, PID) and path planning strategies on a microprocessor (MSP430).
% %       }
% %     }
% %   }

%   \headedsection
%   {Robotics}{\textsc{Beijing, China}}
%   {
%     \headedsubsection
%     {Robot Assisting Shopping for the Handicap, Team Size: 4}{Aug 2011 -- Dec 2011}{

%       \begin{itemize}
%       \item Designed a grasp mechanism that is able to grasp various shapes of
%         items.
        
%       \item Conducted kinematic analysis, strength check, technical
%         drawing (AutoCAD/Solidworks), manufacturing process design and
%         assembling.
        
%       \item The robot conquered all the challenges within 5 minutes and won a
%         runner-up in a campus robot competition.
%       \end{itemize}
%     }
%   }
  
% %   \headedsection
% %   {Game Programming -- \textit{REVERTI} Board Game in C++}{\textsc{Beijing, China}}
% %   {
% %     \headedsubsection
% %     {Team Size: 1}{Feb 2009 -- May 2009}{
% %       \bodytext{A strategy board game for two players programmed by C++.
% %         Functionalities included player registration and log in, ranking systems, and a simple
% %         artificial intelligence (AI).
% %       }
% %     }
% %   }



% %%%%%%%%%%%%%%%%%%%%%%%%%%%%%%%%%%%%%%
%   \spacedhrule{1em}{-0.5em}
%   \roottitle{Honors}
%   \begin{itemize}
%   \item \textbf{Vantage Award - Above \& Beyond}. Process
%     Automation Control \& Optimization Group, Shell Oil Company
%   \item \textbf{Student Travel Award:} 2015 American Control Conference, 2015 Dynamic Systems and Control Conference
%   \end{itemize}

%   \spacedhrule{0.8em}{-0.5em}
%   \roottitle{Service}
%   \begin{itemize}
%   \item \textbf{Invited Reviewer.} 2016 ASME Dynamic Systems
%     and Control Conference
    
%   \item \textbf{Session Organizer.} 2015 ASME Dynamic Systems and Control Conference
%   \end{itemize}

%   \spacedhrule{0.5em}{-0.8em}
%   \roottitle{Technical Talks}
%   \begin{itemize}
%   \item Meal Detection and Estimation for Type 1 Diabetes: A
%       Variable State Dimension Approach. \textit{Oct 2015. 2015 Dynamic Systems \& Control
%       Conference (DSCC), Columbus, OH.}
%   \item Model Predictive Control for Type 1 Diabetes Based on Personalized LTV
%     Model with Insulin and Meal Inputs. \textit{July 2015. 2015 American Control
%       Conference (ACC), Chicago, IL} 
%   \item Fault Detection Based on Modeling and Estimation Methods. \textit{July 2015.
%     Shell Deepwater R\&D Group Lunch \& Learn Seminar. Houston, TX.}
%   \end{itemize}

  
% \spacedhrule{0.5em}{-0.8em}
% \roottitle{Publications}
  
%   \begin{itemize}
%   \item \textbf{Jinyu Xie}, and Qian Wang. ``A Personalized Dietary and Exercise
%     Recommender System Minimizing Clinical Risk for Type 1 Diabetes: An in
%     Silico Study'' \textit{Submitted to 2017 American Control Conference. }
%   \item \textbf{Jinyu Xie}, and Qian Wang. ``A Nonlinear Data-Driven Model of
%     Glucose Dynamics Accounting for Physical Activity for Type 1 Diabetes: An in
%     Silico Study.'' \textit{In Press, Oct 2016. ASME 2016 Dynamic Systems and Control
%     Conference.} 
%   \item \textbf{Jinyu Xie}, and Qian Wang. ``A Variable State Dimension Approach
%     to Meal Detection and Meal Size Estimation: in Silico Evaluation through
%     Basal-Bolus Insulin Therapy for Type 1 Diabetes.'' \textit{In Press. IEEE
%     Transactions on Biomedical Engineering.}
%   \item \textbf{Jinyu Xie}, and Qian Wang. ``Meal Detection and Meal Size Estimation for Type 1 Diabetes Treatment: A Variable State Dimension Approach.'' \textit{In ASME 2015 Dynamic Systems and Control Conference, pp. V001T15A003. American Society of Mechanical Engineers, 2015.}
%   % \item Qian Wang, \textbf{Jinyu Xie}, et al. ``Model Predictive Control for Type 1 Diabetes Based on Personalized Linear Time-Varying Subject Model Consisting of Both Insulin and Meal Inputs An in Silico Evaluation'' \textit{Journal of diabetes science and technology (2015): 1932296815586426.}
%   \item Qian Wang, \textbf{Jinyu Xie}, et al. ``Model Predictive Control for Type 1 Diabetes Based on Personalized Linear Time-Varying Subject Model Consisting of both Insulin and Meal Inputs: an in Silico Evaluation.'' \textit{American Control Conference, 2015: 5782-5787.}
%   \item Qian Wang, Peter Molenaar, Saurabh Harsh, Kenneth Freeman, \textbf{Jinyu Xie}, et al. ``Personalized State-space Modeling of Glucose Dynamics for Type 1 Diabetes Using Continuously Monitored Glucose, Insulin Dose, and Meal Intake An Extended Kalman Filter Approach.'' \textit{Journal of diabetes science and technology 8.2 (2014): 331-345.}
%   \end{itemize}

\end{document}